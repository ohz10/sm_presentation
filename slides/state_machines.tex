\documentclass[aspectratio=169]{beamer}
\usetheme{Madrid}
\usecolortheme{seagull}

\title{State Machines in Rust}
\author{Austin Gilbert}
\begin{document}

\begin{frame}
\begin{center}
\begin{Huge}
State Machines in Rust
\end{Huge}
\end{center}
\end{frame}

\begin{frame}{Disclaimers}
\begin{itemize}
\item I am not representing anyone but myself, my views are my own
\item I am not (yet) a Rust expert
\item I am a C++ expert and Programming Languages nerd exploring Rust
\item I am not presenting production code
\item These are contrived examples meant to demonstrate techniques
\end{itemize}
\end{frame}

\begin{frame}{Overview}
\begin{itemize}
\item Tennis Kata
\item Super Simple State Machine
\item Type-State Pattern; Compile-Time State Machines
\item Asynchronous State Machine Parser
\end{itemize}
\end{frame}

\section{Tennis Kata}
\begin{frame}
\begin{center}
\begin{Huge}
Tennis Kata
\end{Huge}
\end{center}
\end{frame}

\begin{frame}{What is a Kata?}
\begin{itemize}
\item A simple problem you solve repeatedly
\item Apply different patterns or idioms in each solution
\item Enlighten ourselves about advantages and disadvantages of each technique
\end{itemize}
\end{frame}

\begin{frame}{Tennis Kata}
\begin{block}{Goal}
Model a game of tennis between two players
\end{block}
\begin{block}{Scoring}
\begin{itemize}
\item Love
\item Fifteen
\item Thirty
\item Forty
\item Game
\end{itemize}
\end{block}
\begin{block}{Deuce}
\begin{itemize}
\item both players have 40 points before the game is over
\item then players must win by 2 points
\end{itemize}
\end{block}
\end{frame}

\begin{frame}{Tennis Kata}
\begin{center}
\begin{Huge}The Naive Solution\end{Huge}
\end{center}
\end{frame}

\begin{frame}{Tennis Kata: The Naive Solution}
\begin{center}
\begin{Huge}
CODE
\end{Huge}
\end{center}
\end{frame}

\begin{frame}{Tennis Kata}
\begin{center}
\begin{Huge}The Naive Solution Refactored\end{Huge}
\linebreak
\begin{Large}
Score is now Enum
\end{Large}
\end{center}
\end{frame}

\begin{frame}
\begin{center}
\begin{Huge}
CODE
\end{Huge}
\end{center}
\end{frame}

\begin{frame}{Tennis Kata}
\begin{center}
\begin{Huge}4-state State Machine Solution
\end{Huge}
\end{center}
\end{frame}

\begin{frame}{Tennis Kata: 4-state State Machine}
\begin{center}
\includegraphics[scale=0.45]{sm4.png}
\end{center}
\end{frame}

\begin{frame}{Tennis Kata: 4-state State Machine}
\begin{center}
\begin{Huge}
CODE
\end{Huge}
\end{center}
\end{frame}


\begin{frame}{Tennis Kata}
\begin{Huge}20-state State Machine Solution
\end{Huge}
\end{frame}

\begin{frame}{Tennis Kata: 20-state State Machine}
image
\end{frame}

\section{Super Simple State Machine}
\begin{frame}
\begin{center}
\begin{Huge}
Super Simple State Machine
\end{Huge}
\end{center}
\end{frame}

\begin{frame}{Super Simple State Machine}
image
\end{frame}

\section{Type-State Pattern}
\begin{frame}
\begin{center}
\begin{Huge}
Type-State Pattern
\end{Huge}
\linebreak
\linebreak
\begin{Large}
Compile-Time State Machines
\end{Large}
\end{center}
\end{frame}

\section{Asynchronous State Machine Parser}
\begin{frame}
\begin{center}
\begin{Large}
Asynchronous State Machine Parser
\end{Large}
\end{center}
\end{frame}

\begin{frame}{Asynchronous State Machine Parser}
\begin{block}{Asynchronous here...}
\emph{nothing} to do with \textbf{async} keyword
\end{block}
\begin{block}{Asynchronous here...} comes from the fact we're parsing a byte at a time making this class of parser suitable for parsing chunked network data coming in \emph{asynchronously}
\end{block}
\end{frame}

\begin{frame}{Asynchronous State Machine Parser}
\begin{itemize}
\item Extremely Simple URL Parser
\item Parse URLs for:
	\begin{itemize}
	\item Protocol: {HTTP, HTTPS}
	\item Path
	\item ASCII-only
	\end{itemize}
\item Ignoring:
	\begin{itemize}
	\item Host and Domain Name
	\item Parameters and Arguments
	\item Authorization
	\item Port Information
	\item UTF-8
	\end{itemize}
\end{itemize}
\end{frame}

\begin{frame}{Contact}
\begin{center}
\includegraphics[width=1.5in,height=1.5in]{ball_cap.png}
\includegraphics[width=1.5in,height=1.5in]{email_qr.png}
\includegraphics[width=1.5in,height=1.5in]{ohz_10.signal_qr.jpg}
\end{center}
\end{frame}

\end{document}
